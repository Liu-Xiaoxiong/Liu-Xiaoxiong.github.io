%%%%%%%%%%%%%%%%%%%%%%%%%%%%%%%%%%%%%%%%%
% "ModernCV" CV and Cover Letter
% LaTeX Template
% Version 1.3 (29/10/16)
%
% This template has been downloaded from:
% http://www.LaTeXTemplates.com
%
% Original author:
% Xavier Danaux (xdanaux@gmail.com) with modifications by:
% Vel (vel@latextemplates.com)
%
% License:
% CC BY-NC-SA 3.0 (http://creativecommons.org/licenses/by-nc-sa/3.0/)
%
% Important note:
% This template requires the moderncv.cls and .sty files to be in the same 
% directory as this .tex file. These files provide the resume style and themes 
% used for structuring the document.
%
%%%%%%%%%%%%%%%%%%%%%%%%%%%%%%%%%%%%%%%%%

%----------------------------------------------------------------------------------------
%	PACKAGES AND OTHER DOCUMENT CONFIGURATIONS
%----------------------------------------------------------------------------------------

\documentclass[11pt,a4paper,sans]{moderncv} % Font sizes: 10, 11, or 12; paper sizes: a4paper, letterpaper, a5paper, legalpaper, executivepaper or landscape; font families: sans or roman

\moderncvstyle{casual} % CV theme - options include: 'casual' (default), 'classic', 'oldstyle' and 'banking'
\moderncvcolor{blue} % CV color - options include: 'blue' (default), 'orange', 'green', 'red', 'purple', 'grey' and 'black'


\usepackage{lipsum} % Used for inserting dummy 'Lorem ipsum' text into the template
\usepackage{etaremune}
\usepackage[scale=0.75]{geometry} % Reduce document margins
%\setlength{\hintscolumnwidth}{3cm} % Uncomment to change the width of the dates column
%\setlength{\makecvtitlenamewidth}{10cm} % For the 'classic' style, uncomment to adjust the width of the space allocated to your name
\definecolor{cvblue}{RGB}{0,117,210}
\usepackage{xeCJK}
\setCJKmainfont{STXihei}
%----------------------------------------------------------------------------------------
%	NAME AND CONTACT INFORMATION SECTION
%----------------------------------------------------------------------------------------

\firstname{Xiaoxiong} % Your first name
\familyname{Liu} % Your last name

% All information in this block is optional; comment out any lines you don't need
\title{Curriculum Vitae}
\photo[70pt][0.4pt]{../img/profile-img.jpg} % The first bracket is the picture height; the second is the thickness of the frame around 

%----------------------------------------------------------------------------------------

\begin{document}


%----------------------------------------------------------------------------------------
%	CURRICULUM VITAE
%----------------------------------------------------------------------------------------

\makecvtitle % Print the CV title

%----------------------------------------------------------------------------------------
%	EDUCATION SECTION
%----------------------------------------------------------------------------------------
\section{General}
\cvitem{Born}{2nd Feb 1993}
\cvitem{Nationality}{China}
\cvitem{E-mail}{xxliu@physik.uzh.ch}
\cvitem{Position}{Ph.D. student in condensed matter physics, University of Zurich, Switzerland}

\section{Education}

\cventry{2019--present}{Ph.D. in Condensed Matter Physics}{University of Zurich}{Switzerland}{}{Thesis: Simulation of Nonlinear Electronic Transport Using Wannier Interpolation. Supervisor: Stepan S. Tsirkin, Group leader: Titus Neupert}  % Arguments not required can be left empty

\cventry{2016--2019}{M.Sc. in Theoretical Physics}{Lanzhou University}{China}{}{Thesis: Electronic Properties Study of Topological Weyl Semimetals in Space Group $F\bar{4}3m$. Supervisor: Jianbo Deng}  

\cventry{2012--2016}{B.Sc. in Physics}{Lanzhou University}{China}{}{Thesis: First-principle Investigations of 3d Transition Metal (Fe, Cu, and Co)-doped Rocksalt MgO by Chain. Supervisor: Jianbo Deng}  

\section{Teaching Assistant}
\subsection{University of Zurich}
\cvitem{2022}{Quantum Mechanics}
\cvitem{2022}{Machine Learning for the Sciences}
\cvitem{2021}{Mathematical Methods of Physics I}
\cvitem{2021}{Linear Algebra II}
\cvitem{2020}{Linear Algebra I}
\cvitem{2020}{Scientific Computing}
\subsection{Lanzhou University}
\cvitem{2016}{Mechanics}

\section{Awards}

\cvitem{2018}{National Scholarship of Graduated Student.}

\newpage
\section{Languages}

\cvitemwithcomment{Chinese}{Native}{}
\cvitemwithcomment{English}{Professional Fluency}{}

\section{Find Me on Web}
\cvitem{HomePage}{\href{https://liu-xiaoxiong.github.io/index.html}{\textcolor{gray}{https://liu-xiaoxiong.github.io/index.html} } }
\cvitem{GroupPage}{\href{https://www.physik.uzh.ch/en/groups/neupert/team/Xiaoxiong-Liu.html}{\textcolor{gray}{https://www.physik.uzh.ch/en/groups/neupert/team/Xiaoxiong-Liu.html} } }
\cvitem{Scholar}{\href{https://scholar.google.com/citations?user=s2Py778AAAAJ&hl=zh-CN&oi=ao}{\textcolor{gray}{https://scholar.google.com/citations?user=s2Py778AAAAJ\&hl=zh-CN\&oi=ao} } }
\cvitem{ResearchGate}{\href{https://www.researchgate.net/profile/Xiaoxiong-Liul}{\textcolor{gray}{https://www.researchgate.net/profile/Xiaoxiong-Liu} } }
\cvitem{Github}{\href{https://github.com/Liu-Xiaoxiong}{\textcolor{gray}{https://github.com/Liu-Xiaoxiong} } }
%\cvitem{Gitlab}{\href{https://gitlab.com/Xiaoxiong\_Liu}{\textcolor{gray}{https://gitlab.com/Xiaoxiong\_Liu} } }


\section{Development of Scientific Software (open source)}
\subsection{Author of:}
\cvitem{symmetrize wann matrix}{This code aimed to symmetrize matrix elements from Wannier90. E.g., Hamiltonian and position elements.}
\cvitem{}{Code available at: \href{https://github.com/Liu-Xiaoxiong/symmetrize_wann_matrix}{\textcolor{gray}{https://github.com/Liu-Xiaoxiong/symmetrize\_wann\_matrix}} }

\subsection{Co-Developer of:}
\cvitem{WannierBerri}{An advanced tool for Wannier interpolation and integration of quantities related to Berry curvature and magnetic moment. \href{http://wannier-berri.org}{\textcolor{gray}{http://wannier-berri.org}}}
\cvitem{}{Code available at: \href{https://github.com/wannier-berri/wannier-berri}{\textcolor{gray}{https://github.com/wannier-berri/wannier-berri}} }

%\subsection{Contributor of: (in progress)}
%\cvitem{ASE}{The Atomic Simulation Environment (ASE) is a set of tools and Python modules for setting up, manipulating, running, visualizing, and analyzing atomistic simulations.  I am improving the Wannier function part of ASE. \href{https://wiki.fysik.dtu.dk/ase/}{\textcolor{gray}{https://wiki.fysik.dtu.dk/ase/}}}

\section{Computer Skills}
\cvitemwithcomment{Programing}{}{Python3, Fortran, Mathematica, Linux}
\cvitemwithcomment{DFT code}{}{VASP, QuantumEspresso,FPLO,Abnit,Siesta,ASE}
\cvitemwithcomment{Post-DFT}{}{Wannier90,WannierBerri,WannierTools,Irrep,Z2Pack}
\cvitemwithcomment{High-Throughput}{}{AiiDA}

\section{Publication Activity}
\cvitem{}{17 Publications, including: Nature Material(1), Communications Physics(1), PRL(2), PRB(4), APL(1)}
\cvitem{}{453 citations.}
\cvitem{}{h-index  8}
%----------------------------------------------------------------------------------------
%	AWARDS SECTION
%----------------------------------------------------------------------------------------

\newpage





\section{References}
\cventry{Group Leader}{Titus Neupert}{}{Institut-Physik, University of Zurich, <neupert@physik.uzh.ch>}{}{}
\cventry{Supervisor}{Stepan S. Tsirkin}{}{Centro de Física de Materiales, Universidad del País Vasco,  <stepan.tsirkin@ehu.eus>}{}{}
\cventry{Collaborator}{Ivo Souza}{}{Centro de Física de Materiales, Universidad del País Vasco <ivo\_souza@ehu.eus>}{}{}


\section{Participation in Conferences}
\subsection{Oral Contributions}
\begin{etaremune}
  \item Ab initio calculations of electrical magnetochiral anisotropy with Wannier interpolation, \textbf{APS March meeting}, Las Vegas, USA, March 5-10, 2023
  \item Symmetrization of berry curvature and magnetic moment, \textbf{Wannier 2022 Developers Meeting (smr 3757)}, ICTP, Trieste, Italy, May 23-27, 2022
  \item Gauge-covariant derivatives of the Berry curvature and orbital moment by Wannier interpolation, \textbf{APS March meeting}, Virtual, USA, March 15-19, 2021
\end{etaremune}

\subsection{Poster Contributions}
\begin{etaremune}
  \item Ab initio calculations of electrical magnetochiral anisotropy with Wannier interpolation, \textbf{Swiss Workshop on Materials with Novel Electronic Properties Basic research and applications}, Les Diablerets, Switzerland, August 29-31, 2022
  \item Ab initio calculations of electrical magnetochiral anisotropy with Wannier interpolation, \textbf{Psi-K Conference}, EPFL, Lausanne, Switzerland, August 22-25, 2022
  \item Systematic study of magnetotransport responses with Berry-Boltzmann formalism, \textbf{First-Principles Modelling of Defects in Solids Workshop}, ETHz, Zurich, Switzerland, June 13-15, 2022
  \item Systematic study of magnetotransport responses with Berry-Boltzmann formalism, \textbf{Wannier 2022 Summer School}, ICTP, Trieste, Italy May 16-20, 2022
  \item Wannier Interpolation of Berry-Boltzmann Formalism for Berry Curvature related quantities with WannierBerri, \textbf{Condensed Matter Theory Symposium}, ETHz, Zurich, Switzerland, September 22, 2021
  \item Gauge-covariant derivatives of the Berry curvature and orbital moment by Wannier interpolation, \textbf{Virtual DPG Spring Meeting}, Virtual, Germany, March 1-4, 2021
  \item Gauge-covariant derivatives of the Berry curvature and orbital moment by Wannier interpolation, \textbf{20th International Workshop on Computational Physics and Materials Science: Total Energy and Force Methods}, Virtual, Italy, February 23-25, 2021
  \item Gauge-covariant derivatives of the Berry curvature and orbital moment by Wannier interpolation, \textbf{Virtual Electronic Structure Workshop}, Virtual, USA, June 3-5, 2020
\end{etaremune}

\newpage

\section{Publications}

\begin{etaremune}
  \setlength{\itemsep}{5pt}
  \item Electrical magnetochiral anisotropy in trigonal tellurium from first principles\\ \textcolor{cvblue}{Xiaoxiong Liu}, Ivo Souza, Stepan S. Tsirkin, \textbf{arXiv e-prints}, arXiv:2303.10164
  \item Covariant derivatives of Berry-type quantities: Application to nonlinear transport\\ \textcolor{cvblue}{Xiaoxiong Liu}, Stepan S. Tsirkin, Ivo Souza, \textbf{arXiv e-prints}, arXiv:2303.10129
   \item Origin of spin reorientation and intrinsic anomalous Hall effect in the kagome ferrimagnet TbMn6Sn6\\ DC Jones, S Das, H Bhandari, \textcolor{cvblue}{Xiaoxiong Liu}, P Siegfried, MP Ghimire, SS Tsirkin, II Mazin, NJ Ghimire, \textbf{arXiv e-prints}, arXiv: 2203.17246
  \item Emergent Edge Modes in Shifted Quasi-One-Dimensional Charge Density Waves\\ SB Zhang, \textcolor{cvblue}{Xiaoxiong Liu}, MS Hossain, JX Yin, MZ Hasan, T Neupert, \textbf{Physical Review Letters}  130, 106203, (2023)
   \item Triple nodal points characterized by their nodal-line structure in all magnetic space groups\\ PM Lenggenhager, \textcolor{cvblue}{Xiaoxiong Liu}, T Neupert, T Bzdušek,  \textbf{Physical Review B} 106 (8), 085128, (2022), (共同通信)
  \item Universal higher-order bulk-boundary correspondence of triple nodal points\\ PM Lenggenhager, \textcolor{cvblue}{Xiaoxiong Liu}, T Neupert, T Bzdušek, \textbf{Physical Review B} 106 (8), 085129, (2022)
  \item Intriguing magnetism of the topological kagome magnet $TbMn_6Sn_6$\\ C Mielke III, Wenlong Ma, V Pomjakushin, O Zaharko, \textcolor{cvblue}{Xiaoxiong Liu}, J-X Yin, SS Tsirkin, TA Cochran, M Medarde, V Poree, D Das, CN Wang, J Chang, T Neupert, A Amato, S Jia, MZ Hasan, H Luetkens, Z Guguchia, \textbf{Communications Physics} 5 (1), 1-9 (2022)
  \item Signatures of Weyl fermion annihilation in a correlated kagome magnet\\ I. Belopolski, T. A. Cochran, \textcolor{cvblue}{Xiaoxiong Liu}, Z. Cheng, X. Yang, Z. Guguchia, S. S. Tsirkin, J. Yin, P. Vir, G. S. Thakur, S. Zhang, J. Zhang, K. Kaznatcheev, G. Cheng, G. Chang, D. Multer, N. Shumiya, M. Litskevich, E. Vescovo, T. K. Kim, C. Cacho, N. Yao, C. Felser, T. Neupert, M. Z. Hasan, \textbf{Physical Review letters} 127 (25), 256403, (2021), (共同一作)
  \item Unconventional chiral charge order in kagome superconductor KV3Sb5\\ Y. Jiang, J. Yin, M. M. Denner, N. Shumiya, B. R. Ortiz, G. Xu, Z. Guguchia, J. He, M. S. Hossain, \textcolor{cvblue}{Xiaoxiong Liu}, J. Ruff, L. Kautzsch, S. Zhang, G. Chang, I. Belopolski, Q. Zhang, T. A. Cochran, D. Multer, M. Litskevich, Z. Cheng, X. Yang, Z. Wang, R. Thomale, T. Neupert, S. D. Wilson, M. Z. Hasan, \textbf{Nature Materials} 20 (10), 1353-1357, (2021)
  \item From triple-point materials to multiband nodal links\\ PM Lenggenhager, \textcolor{cvblue}{Xiaoxiong Liu}, SS Tsirkin, T Neupert, T Bzdušek, \textbf{Physical Review B} 103 (12), L121101, (2021)
  \item Magneto-transport and Shubnikov–de Haas oscillations in the type–II Weyl semimetal candidate NbIrTe4 flake\\ X. Huang, \textcolor{cvblue}{Xiaoxiong Liu}, P. Yu, P. Li, J. Cui, J. Yi, J. Deng, J. Fan, Z. Ji, F. Qu, X. Jing, C. Yang, L Lu, Z. Liu, G. Liu, \textbf{Chinese Physics Letters} 36 (7), 077101, (2019), (共同一作)
  \item Quantum anomalous Hall effect and topological phase transition in two-dimensional antiferromagnetic Chern insulator NiOsCl6\\ WW Yang, L Li, JS Zhao, \textcolor{cvblue}{Xiaoxiong Liu}, JB Deng, XM Tao, XR Hu, \textbf{Journal of Physics: Condensed Matter} 30 (18), 185501, (2018)
  \item A nonmagnetic topological Weyl semimetal in quaternary Heusler compound CrAlTiV\\ \textcolor{cvblue}{Xiaoxiong Liu}, L Li, Y Cui, J Deng, X Tao, \textbf{Applied Physics Letters} 111 (12), 122104, (2017)
  \item Ternary Weyl semimetal $NbIrTe_4$ proposed from first-principles calculation\\ L Li, HH Xie, JS Zhao, \textcolor{cvblue}{Xiaoxiong Liu}, JB Deng, XR Hu, XM Tao, \textbf{Physical Review B} 96 (2), 024106, (2017)
  \item First-principle investigations of 3d transition metal (Fe, Cu, and Co)-doped rocksalt MgO by chain\\ \textcolor{cvblue}{Xiaoxiong Liu}, Q Gao, L Li, J Zhao, X Hu, J Deng, \textbf{Journal of Superconductivity and Novel Magnetism} 30 (6), 1635-1641, (2017)
  \item Effect of As and Nb doping on the magnetic properties for quaternary Heusler alloy FeCoZrGe\\ GY Mao, \textcolor{cvblue}{Xiaoxiong Liu}, Q Gao, L Li, HH Xie, G Lei, JB Deng, \textbf{Journal of Magnetism and Magnetic Materials} 398, 1-6, (2016)
  \item First-principle study of half-metallic ferromagnetism in rocksalt XO (X= Li, K, Rb, Cs)\\ G Lei, \textcolor{cvblue}{Xiaoxiong Liu}, HH Xie, L Li, Q Gao, JB Deng,\textbf{Journal of Magnetism and Magnetic Materials} 397, 176-180, (2016)

\end{etaremune}









%----------------------------------------------------------------------------------------

\end{document}